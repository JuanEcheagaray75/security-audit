\documentclass{article}
\usepackage[utf8]{inputenc}
\usepackage[spanish]{babel}
\usepackage[]{amsthm}
\usepackage{amsmath}
\usepackage[]{amssymb}
\usepackage{graphicx}
\usepackage{wrapfig}
\usepackage[letterpaper, margin=1.5in]{geometry}
\usepackage[hidelinks]{hyperref}
\usepackage{pdfpages}
\decimalpoint


\begin{document}

    % Portada
    \begin{titlepage}
        \begin{center}
            \begin{figure}
                \centering
                \includegraphics[scale=0.13]{../../../logo_itesm.png}\\ % Logo de la institución
            \end{figure}
        \vspace{5cm}
        \LARGE{Instituto Tecnológico y de Estudios Superiores de Monterrey}\\
        \fontsize{12}{14}\selectfont
        \vspace{1cm}
        \textbf{ Actividad 2.1. Cifrado César, sustitución monoalfabética y Vigenère} \\ % Nombre de la tarea
        \vspace{0.7cm}
        Juan Pablo Echeagaray González \\ % Nombre de autor 1
        \vspace{0.2cm}
        A00830646 \\ % Matrícula autor 1
        \vspace{0.7cm}
        Análisis de Criptograrfía y Seguridad\\ % Materia
        \vspace{0.2cm}
        MA2002B.300 \\ % Clave de la materia
        \vspace{0.2cm}
        Dr.-Ing. Jonathan Montalvo-Urquizo \\ % Nombre del profesor
        \vspace{0.7cm}
        26 de mayo del 2022 \\ % Fecha de entrega
        \end{center}
    \end{titlepage}

    \section{Sobre el cifrado César}
        El cifrado de César fue el más sencillo de implementar de todos los métodos vistos, su tiempo de encriptado y desencriptado (aún sin conocer la llave) es mínusculo a comparación de los demás. La única barrera encontrada es la incertidumbre del lenguaje original del \emph{plaintext}. Para la implementación sugerida solamente se toman en cuenta caracteres de lenguas romances, lo cual no nos asegura poder realizar un proceso de criptoanálisis en una situación general.

        El proceso de encriptado y desencriptado con una llave conocida le tomó a la máquina de prueba un promedio de $21 \mu s$; cuando no disponía de una llave se realizó un ataque por fuerza bruta, el tiempo de desencriptado incrementó a $278 \mu s$. El incremento de tiempo no es tan grande porque solamente se deben de probar un total de 26 diferentes llaves para desencriptar el mensaje; al final de este proceso se necesita de un ser humano que rectifique los resultados.

    \section{Sobre el cifrado monoalfabético}
        El cifrado monoalfabético es mucho más fuerte que un simple cifrado César, se pasa de un espacio de $n$ posibles llaves a $n!$, un espacio que tiene un tamaño de al menos 25 ordenes de magnitud superior cuando solamente se tiene un alfabeto consistente de letras. 
        
        Enumerar todas las posibles llaves e intentar desencriptar el mensaje podría ser una solución factible en equipos con un mayor poder de cómputo; sin embargo, una aproximación por fuerza bruta de ese estilo no es necesario para romper este cifrado. Dado que este método sigue siendo una simple sustitución, podemos hacer uso de herramientas estadísticas para encontrar el mensaje original, de forma más específica, se realiza un \emph{análisis de frecuencias}.

        Suponiendo que el criptoanalista conozca el lenguaje original del \emph{plaintext}, este puede comenzar el proceso de desencriptado al analizar la frecuencia relativa de cada caracter dentro del \emph{ciphertext}; después se pueden comparar estas frecuencias con algunas tablas de frecuencias de letras en el lenguaje a tratar. En la primera iteración se pueden remplazar las primeras $n$ letras más frecuentes del \emph{ciphertext} con las $n$ primeras letras más frecuentes de las tablas conocidas.

        Después de esta iteración se analiza el texto, con la esperanza de que esta sustitución torne más legibles algunas secciones del \emph{ciphertext}. De aquí, se puede ir determinando de forma empírica las sustituciones a realizar, por ejemplo, si uno encuentra después de la primera iteración el texto \emph{tha} y se sabe que el texto original está escrito en inglés, se propone la sustitución $a \rightarrow e$.

        Este método tomó más tiempo de implementar y romper que el anterior, pero se tornó bastante sencillo de romper una vez que se conocían los remplazos adecuados. Este caso de estudio fue más sencillo de resolver dada la longitud del texto original de alrededor de $48,000$ caracteres. Con la implementación utilizada no existe una manera metódica de determinar el tiempo de ruptura del encriptado.

    \section{Sobre el cifrado Vigenère}
        De los desarrollados en esta entrega, el cifrado de Vigenère es el más fuerte de todos, por mucho tiempo fue considerado irrompible, pero con las tecnologías actuales e inteligentes análisis es posible llegar a romperlo. La fortaleza sobre los anteriores es que entra en una categoría de métodos de cifrado conocidos como \emph{polialfabéticos}, para este caso se utiliza un cifrado de César para cada uno de los caracteres en el \emph{plaintext} que logra obscurecer por completo toda la información estadística del mensaje.

        El algoritmo de encriptado recibe 2 secuencias de caracteres, una consistente del mensaje en claro y otra de la llave usada para cifrar, $m$ es la longitud de la llave. El caracter iésimo del texto cifrado se calcula con la siguiente expresión:

        \begin{equation}
            C_i = (p_i + k_{i\text{ mod } m}) \text{ mod } 26
        \end{equation}

        Para desencriptar el mensaje se tiene una expresión simple también:
        \begin{equation}
            p_i = (C_i - k_{i\text{ mod } m}) \text{ mod } 26
        \end{equation}

        El espacio de búsqueda para la llave se torna más grande, si tomamos $n$ como la longitud del alfabeto del lenguaje, y $m$ la longitud del mensaje, el espacio de búsqueda es entonces $n^m$. No existe una manera clara de intentar resolver este problema por fuerza bruta, ¿Con qué llave empezar? ¿Cuál será su longitud? Son algunas de las preguntas a las que nos enfrentaríamos si enfrentáramos el problema ciegamente.
    
    \clearpage
    \appendix

    \section{Código implementado y resultados}
        \includepdf[pages=-]{classical-ciphers-converted.pdf}

\end{document}
