\documentclass{article}
\usepackage[utf8]{inputenc}
\usepackage[spanish]{babel}
\usepackage[]{amsthm}
\usepackage{amsmath}
\usepackage[]{amssymb}
\usepackage{graphicx}
\usepackage{wrapfig}
\usepackage[letterpaper, margin=1.5in]{geometry}
\usepackage[hidelinks]{hyperref}
\decimalpoint


\begin{document}
    \begin{titlepage}
        \begin{center}
            \begin{figure}
                \centering
                \includegraphics[scale=0.13]{../../../logo_itesm.png}\\ % Logo de la institución
            \end{figure}
        \vspace{5cm}
        \LARGE{Instituto Tecnológico y de Estudios Superiores de Monterrey}\\
        \fontsize{12}{14}\selectfont
        \vspace{1cm}
        \textbf{Actividad 3.1. Crear una red en Packet Tracer}\\ % Nombre de la tarea
        \vspace{0.7cm}
        Juan Pablo Echeagaray González\\ % Nombre de autor 1
        \vspace{0.2cm}
        A00830646\\ % Matrícula autor 1
        \vspace{0.7cm}
        Análisis de Criptografía y Seguridad\\ % Materia
        \vspace{0.2cm}
        MA2003B.300\\ % Clave de la materia
        \vspace{0.2cm}
        Dr. Alberto Francisco Martínez Herrera\\ % Nombre del profesor
        \vspace{0.7cm}
        29 de mayo del 2022\\ % Fecha de entrega
        \end{center}
    \end{titlepage}

    \tableofcontents

    \section{Reporte}

        \subsection{Procedimiento}

            

        \subsection{Resultados}

    \appendix

    \section{Figuras}



\end{document}