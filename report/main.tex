\documentclass[10pt]{article}
% Paquetes
\usepackage[utf8]{inputenc}
\usepackage[spanish]{babel}
\usepackage[]{amsthm}
\usepackage{amsmath}
\usepackage[]{amssymb}
\usepackage{graphicx}
\usepackage{wrapfig}
\usepackage[letterpaper, top=0.78in, bottom=0.78in, left=0.98in, right=0.98in]{geometry}
\usepackage[hidelinks]{hyperref}
\usepackage{csvsimple}
\usepackage{pdflscape}
\decimalpoint

% Comandos personalizados
\renewcommand{\baselinestretch}{1.5}
% 20 mm superior e inferior <=> 0.78in
% 25 mm derecha e izquierda <=> 0.98in

\begin{document}
    \begin{titlepage}
        \begin{center}
            % School logo
            \begin{figure}
                \centering
                \includegraphics[scale=0.13]{../img/logo_itesm.png}\\ % Logo de la institución
            \end{figure}
            \vspace{5cm}
            % School data
            \LARGE{Instituto Tecnológico y de Estudios Superiores de Monterrey}\\
            \vspace{1cm}
            \large Escuela de Ingeniería y Ciencias \\
            \vspace{0.2cm}
            \large Ingeniería en Ciencias de Datos y Matemáticas \\
            \vspace{0.2cm}
            \large Análisis de Criptografía y Seguridad\\
            \vspace{1cm}
            \textbf{Auditoría de Seguridad y Plan de Mitigación: Caso PINK Accesorios y Regalos}\\ % Nombre de la tarea
            \vspace{0.7cm}
            % Tabla de integrantes del equipo
            \begin{table}[h!]
                \centering
                \begin{tabular}{ ||c|c|| }
                    \hline
                    Nombre & Matrícula \\
                    \hline
                    Julio Avelino Amador Fernández & A01276513 \\
                    \hline
                    Juan Pablo Echeagaray González & A00830646 \\
                    \hline
                    Verónica Victoria García De la Fuente & A00830383 \\
                    \hline
                    Erika Martínez Meneses & A01028621 \\
                    \hline
                    Emily Rebeca Méndez Cruz & A00830768 \\
                    \hline
                    Ana Paula Ruiz Alvaro & A01367467 \\
                    \hline
                \end{tabular}
            \end{table}
            \vspace{0.7cm}
            \large Dr. Alberto Francisco Martínez Herrera \\ % Nombre del profesor 1
            \vspace{0.2cm}
            \large Dr.-Ing. Jonathan Montalvo-Urquizo \\ % Nombre del profesor 2
            \vspace{0.2cm}
            \large Socio Formador: Kaspersky \\
            \vspace{0.2cm}
            \large Monterrey, Nuevo León \\
            \vspace{0.2cm}
            \large 30 de mayo del 2022 \\
            \vspace{1cm}
            \footnotesize El trabajo realizado es para fines académicos sin fines de lucro. Queda prohibida la reproducción total o parcial de los datos (en bruto o enmascarados), resultados, modelos y conclusiones sin el previo consentimiento por escrito otorgado por la PyME. \\
        \end{center}
    \end{titlepage}
    
    \tableofcontents
    \clearpage

    \section{Introducción}

    \section{Inventario}

        \subsection{Equipos y software}

        \subsection{Personal de la PyME}

            % Listar todas las personas que tienen acceso a los equipos mencionados en la tabla en conjunto con una descripción del software que utilizan acorde a sus responsabilidades designadas

        \subsection{Topología de la red}

            En la figura \ref{fig:pink-network-topology} se puede apreciar la topología de la red de la PyME. Siempre se encuentran conectados 6 dispositivos a la red. El equipo \emph{Tienda La Central} está a cargo del propietario \emph{Propietario 1}, los smartphones \emph{Smartphone Propietario 1 Central} y \emph{Smartphone Propietario 2 Central} están a cargo de los 2 respectivos propietarios de la empresa consiguiente. Respecto a la PyME a analizar, el equipo \emph{PINK} es usado en conjunto por el \emph{Propietario 2} y el \emph{Empleado N}.

            \begin{figure}[h!]
                \centering
                \includegraphics[scale=0.6]{img/sme-network-topology.png}\\ % Logo de la institución
                \caption{Topología original de la red inventariada para la PyME PINK Accesorios y Regalos}
                \label{fig:pink-network-topology}
            \end{figure}

    \appendix
    % Need to deactivate and activate _ sign, regularly used in software names and versions
    \catcode`\_=12% deactivate $ sign
    % May not need to use landscape after all, gather all the text that will be typed in the report before changing this section
    \section{Inventario de Software}
        % Installed software on both connected machines
        \begin{landscape}
            \begin{table}[h!]
                \centering
                \scalebox{0.47}{
                    \csvreader[tabular = ||c|c|c|c|c||,
                                table head = \hline DisplayName & DisplayVersion & Publisher & InstallDate & id \\ \hline,
                                late after last line = \\ \hline,
                                ]{../inventory/data/processed/inventario-pink-software.csv}{}%
                                {\csvcoli & \csvcolii & \csvcoliii & \csvcoliv & \csvcolv}}
                \caption{Software inventariado de la PyME}
                \label{table:software-pyme}
            \end{table}
        \end{landscape}

    % Information on the machines and OS
    \section{Inventario de equipos}
        % Information on the machines connected to the same network


    \catcode`\_=3% reactivate $ sign


\end{document}