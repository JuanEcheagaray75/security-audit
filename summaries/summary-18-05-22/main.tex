\documentclass[12pt]{article}
\usepackage[utf8]{inputenc}
\usepackage[spanish]{babel}
\usepackage[]{amsthm}
\usepackage{amsmath}
\usepackage[]{amssymb}
\usepackage{graphicx}
\usepackage{wrapfig}
\usepackage[letterpaper, margin=1.5in]{geometry}
\usepackage[hidelinks]{hyperref}
\decimalpoint


\begin{document}

    \begin{titlepage}
        \begin{center}
            \begin{figure}
                \centering
                \includegraphics[scale=0.13]{../../logo_itesm.png}\\ % Logo de la institución
            \end{figure}
        \vspace{5cm}
        \LARGE{Instituto Tecnológico y de Estudios Superiores de Monterrey}\\
        \fontsize{12}{14}\selectfont
        \vspace{1cm}
        \textbf{Participación con Socio Formador, Miércoles 18 de Mayo de 2022}\\ % Nombre de la tarea
        \vspace{0.7cm}
        Juan Pablo Echeagaray González\\ % Nombre de autor 1
        \vspace{0.2cm}
        A00830646\\ % Matrícula autor 1
        \vspace{0.7cm}
        Análisis de Criptografía y Seguridad\\ % Materia
        \vspace{0.2cm}
        MA2002B.300\\ % Clave de la materia
        \vspace{0.2cm}
        Dr. Alberto F. Martínez \\ % Nombre del profesor
        \vspace{0.2cm}
        Dr.-Ing. Jonathan Montalvo-Urquizo \\
        \vspace{0.7cm}
        22 de mayo del 2022\\ % Fecha de entrega
        \end{center}
    \end{titlepage}
   

    % \begin{enumerate}
    %     \item Building out a threat intelligence without action is about as valuable as not having inteligence at all
    %     \item Tecnologías de la información para incrementar el potencial del negocio, hacerlo crecer; la ciberseguridad en cambio se debe de encargar de cuidarle la espalda al negocio
    %     \item No existen estrategias de defensa genérica
    %     \item Buena presentación de la información, clara, concisa y puntual, lograr que los stake-holders comprendan nuestro análisis y la importancia del mismo
    %     \item La ciberseguridad es un campo dinámico, las estrategias implementadas ayer podrían no defenderme de las amenazas del mañana, siempre se debe de estar trabajando en el fortalecimiento de la arquitectura de nuestra red
    %     \item Defensa en profundidad, aka. esconder tus cosas por capas, tus activos más valiosos deben de estar en la capa más profunda
    %     \item Un simple respaldo de la información después de un ataque no es la mejor manera de actuar, una vez que hemos sido vulnerados no basta con llevar a la empresa a como estaba antes por medio de un respaldo, se debe de encontrar la vulnerabilidad y parcharla
    %     \item Detection, Threat intelligence, Assessment and Engineering, Adversary Emulation
    %     \item pasar de una postura reactiva a una preventiva
    % \end{enumerate}

\end{document}